	\documentclass{article}
	\pagenumbering{gobble}
	\usepackage[utf8]{inputenc}
	\usepackage{amsmath}
	\usepackage{amssymb}
	\setlength{\parindent}{0pt}
	\begin{document}

		\begin{itemize}
			\item Records are identified by their subscripts $ i $, such as ($ i \in \mathbb{N} $). Notation: $ r_{i} $.
		\end{itemize}

		\newpage

		\begin{equation} \label{eqn1}
			alpha = \sum_{i=1}^{10} r_{i}
		\end{equation}

		\newpage

		\begin{itemize}
			\item Big records are identified by their subscripts $ i $, such as ($ i \in \mathbb{N} $). Notation: $ R_{i} $.
		\end{itemize}

		\newpage

		\begin{equation} \label{eqn2}
			beta = \sum_{i=1}^{10} R_{i}
		\end{equation}

	\end{document}