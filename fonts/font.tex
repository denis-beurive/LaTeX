\documentclass{basic}

\usepackage{showframe}

% The line below is important.
% By default, the beginnings of paragraphs are indented.
\setlength{\parindent}{0pt}

% The line below sets the length of the space between the \framebox edges and its content.
% \framebox boxes are used to draw edged around the \parbox boxes.
\setlength{\fboxsep}{0px}

% This package allows the selection of any font installed on your system (only if you use lualatex).
% Find the names of all fonts under Ubuntu:
% find /usr/share/texmf/fonts/ -name "*.otf" | xargs -n1 fc-scan --format "%{fullname}\n" | sed 's/,.\+$//'
\usepackage{fontspec}
\setmainfont[Ligatures=TeX]{Gentium}

% Font setting can be done through commands.
% Below, we define 2 commands that set 2 fonts ("LMRoman6-Regular" and "LMMonoLt10-Oblique").
% Be aware that, used without care, these commands change the font all the way down the end of the document.
% You often want to restrict the use of a font to a certain area. You can do so by using "{...}" or "\begingroup ... \endgroup".

\newcommand{\myfontOne}{
       \setmainfont[Ligatures=TeX]{LMRoman6-Regular}
}

\newcommand{\myfontTwo}{
       \setmainfont[Ligatures=TeX]{LMMonoLt10-Oblique}
}

% Or the fonts can be set through the use of environments.
% Please note that this technique isolates the modification introduced by the selection of the font to the area defined by the environment.

\newenvironment{myfontOneEnv}{
       \setmainfont[Ligatures=TeX]{LMRoman6-Regular}
}

\newenvironment{myfontTwoEnv}{
       \setmainfont[Ligatures=TeX]{LMMonoLt10-Oblique}
}

% At last, you can use \DeclareTextFontCommand.
% This solution is very interesting from inline font changing.

\DeclareTextFontCommand{\myFontOneFt}{\myfontOne}

\DeclareTextFontCommand{\myFontTwoFt}{\myfontTwo}

\begin{document}

     Default main font.

     % Please note the use of the enclosing characters "{" and "}".
     % This construct is used to isolate the effect of the command to the area delimited by the characters "{" and "}".
     {\myfontOne The first defined font.}

     Default main font.

     % Please note the use of the enclosing characters "{" and "}".
     % This construct is used to isolate the effect of the command to the area delimited by the characters "{" and "}".
     {\myfontTwo The second defined font.}

     Default main font.

     % The same result can be obtained by using the construct \begingroup ... \endgroup.
    
     \begingroup
          \myfontOne
          The first defined font.         
     \endgroup 

     Default main font.

     \begingroup
          \myfontTwo
          The second defined font.
     \endgroup 

     Default main font.

     % The same result can also be obtained by using environments.

     \begin{myfontOneEnv}
          The first defined font.         
     \end{myfontOneEnv}

     Default main font.

     \begin{myfontTwoEnv}
          The second defined font.         
     \end{myfontTwoEnv}

     Default main font.

     % And, the other solution.

     ==> \myFontOneFt{The first defined font.}

     Default main font.

     ==> \myFontTwoFt{The second defined font.}

     Default main font.

\end{document}


